\documentclass[11pt, oneside]{article}   	% use "amsart" instead of "article" for AMSLaTeX format
\usepackage{geometry}                		% See geometry.pdf to learn the layout options. There are lots.
\geometry{letterpaper}                   		% ... or a4paper or a5paper or ... 
%\geometry{landscape}                		% Activate for for rotated page geometry
%\usepackage[parfill]{parskip}    		% Activate to begin paragraphs with an empty line rather than an indent
\usepackage{graphicx}				% Use pdf, png, jpg, or eps with pdflatex; use eps in DVI mode
								% TeX will automatically convert eps --> pdf in pdflatex		
\usepackage{amssymb}
\usepackage{amsmath}

\title{Transfer Theorems}
\author{Carl Raymond}
%\date{}							% Activate to display a given date or no date

\begin{document}
\maketitle

Once the generating function is known, extract the coefficients using one of the transfer theorems.

\subsection*{Taylor's Theorem}
If $f(z)$ has $N$ derivatives, then $[z^N]f(z) = f^{(N)}(0)/N!$.

\subsection*{Rational Functions Transfer Theorem}
If $f(z)$ and $g(z)$ are polynomials, then
\[
	[z^N]\frac{f(z)}{g(z)} = - \frac{\beta f(1/\beta)}{g'(\beta)} \beta^N
\]
where $\beta$ is the largest root of $g(z)$ (provided that it has multiplicity 1).

\subsection*{Radius-of-Convergence Transfer Theorem}
If $f(z)$ has radius of convergence $> 1$ with $f(1) \ne 0$, then
\[
	\left[ z^N \right] \frac{f(z)}{(1-z)^\alpha} \sim f(1)\binom{n+\alpha-1}{n} \sim \frac{f(1)}{\Gamma(\alpha)} n^{\alpha-1}
\]

\subsection*{Exercise 5.1}
How many bitstrings of length $N$ have no $000$?

A bitstring with no $000$, $B_{000}$, is either empty, or 0, or 00, or it is 1 followed by some $b \in B_{000}$, or 01 followed by some  $b \in B_{000}$,
or 001 followed by some $b \in B_{000}$.  Symbolically,
\[
	\mathcal{B} = \emptyset   + 0 + 00 + 1 \times \mathcal{B} + 01 \times \mathcal{B} + 001 \times \mathcal{B}
\]
The cost of a bitstring is its length, and so the generating function is $z^n$. The generating function satisfies
\begin{align*}
	B(z) &= 1 + z + z^2 + zB(z) + z^2B(z) + z^3B(z) \\
		&= \frac{1+z+z^2}{1-z-z^2-z^3}
\end{align*}

The first few terms of the expansion are $B(z) = 1 + 2z + 4z^2 + 7z^3 + 13z^4 + 24z^5 + 44z^6 + 81z^7 + 149z^8 + 274z^9 + O(z^{10})$.


\subsection*{Exercise 5.3}
Let $\mathcal{U}$ be the set of binary trees with the size of the tree defined to be the total number of nodes (internal plus external), so that
the generating function of the counting sequence is $U(z) = z + z^3 + 2z^5 + 5z^7 + 14z^9 + \cdots$. Derive an explicit expression for $U(z)$.

This is like the previous derivation of the number of trees except that the size of an external node is 1 instead of 0. Denote an external node
with $\circ$, where $\left| \circ \right| = 1$. Denote an internal node with $\bullet$, where $\left| \bullet \right| = 1$.  The generating functions for
both is $z$. The definition of a tree is that a tree is empty, or it consists of an internal node with some tree as the left child, and another tree as the right child.
Symbolically,
\[
	\mathcal{U} = \circ + \bullet \times \mathcal{U} \times \mathcal{U}
\]
Using the generating functions for $\circ$ and $\bullet$,
\[
	U(z) = z + zU^2(z)
\]
Rearrange to $z - U(z) + zU^2(z) = 0$ and apply the quadratic formula where $a=z$, $b=-1$ and $c=z$ to obtain
\[
	U(z) = \frac{1-\sqrt{1-4z^2}}{2z}
\]
Wolfram Alpha confirms this is $z + z^3 + 2z^5 + 5z^7 + 14z^9 + 42z^{11} + O(z^{13})$.

\subsection*{Exercise 5.7}
Derive an EGF for the number of permutations whose cycles are all of odd length.

\subsection*{Exercise 5.15}
Find the average number of internal nodes in a binary tree of size $n$ with both children internal

\subsection*{Exercise 5.16}
Find the average number of internal nodes in a binary tree of size $n$ with one child internal and one child external.

\end{document}  