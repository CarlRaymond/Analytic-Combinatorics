\documentclass[11pt, oneside]{article}   	% use "amsart" instead of "article" for AMSLaTeX format
\usepackage{geometry}                		% See geometry.pdf to learn the layout options. There are lots.
\geometry{letterpaper}                   		% ... or a4paper or a5paper or ... 
%\geometry{landscape}                		% Activate for for rotated page geometry
%\usepackage[parfill]{parskip}    		% Activate to begin paragraphs with an empty line rather than an indent
\usepackage{graphicx}				% Use pdf, png, jpg, or eps with pdflatex; use eps in DVI mode
								% TeX will automatically convert eps --> pdf in pdflatex		
\usepackage{amssymb}
\usepackage{amsmath}

\title{Chapter 3 Notes}
\author{Carl Raymond}
%\date{}							% Activate to display a given date or no date

\begin{document}
\maketitle
\section*{Exercises}
\subsection*{Exercise 3.20}

Solve the recurrence twice, with different initial conditions.
\[
	a_n = 3a_{n-1} - 3a_{n-2} + a_{n-3} \quad \text{for $n>2$ with}  \left\{
	\begin{array}{l l}
		\textrm{ $a_0=a_1=0$; $a_2=1$} \\
		\textrm{ $a_0=0$; $a_1 = a_2 = 1$}
	\end{array} \right.
\]

\begin{tabular}{r | r | r}
$n$		& $a_n$	& $a_n$ \\
\hline 
0		& 0		& 0 \\
1		& 0		& 1 \\
2		& 1		& 1 \\
3		& 3		& 0 \\
4		& 6		& -2 \\
5		& 10		& -5 \\
6		& 15		& -9
\end{tabular}

\textbf{First conditions:}
Write an equation valid for $n \ge 0$.
\[
	a_n = 3a_{n-1} - 3a_{n-2} + a_{n=3} + \delta_{n2}
\]

Form the generating function
\[
	A(z) = \sum_{k \ge 0} a_k z^k = 3zA(z) - 3z^2A(z) + z^3A(z) + z^2
\]
\[
	A(z) \left[ 1 -3z + 3z^2 -z^3 \right] = z^2
\]
\[
	A(z) = \frac{z^2}{1-3z+3z^2-z^3}
\]

Since $1-3z+3z^2-z^3 = (1-z)^3$, by partial fraction expansion
\[
	A(z) = \frac{1}{1-z} - \frac{2}{(1-z)^2}  + \frac{1}{(1-z)^3}
\]

Extract coefficients for a closed form for $a_n$:
\begin{align*}
	a_n &= \left[z^n\right] \frac{1}{1-z} - \left[z^n\right] \frac{2}{(1-z)^2} + \left[z^n\right] \frac{1}{(1-z)^3} \\
		&= 1 - 2(n+1) + \frac{(n+1)(n+2)}{2} \\
		&= \frac{n(n-1)}{2}
\end{align*}

\textbf{Second conditions:} Write an equation valid for all $n \ge 0$:
\[
	a_n = 3a_{n-1} - 3a_{n-2} + a_{n-3} + \delta_{n1} - 2\delta_{n2}
\]
Form the generating function
\[
	A(z) = \sum_{k \ge 0} a_k z^k = 3zA(z) - 3z^2A(z) + z^3A(z) + z - 2z^2
\]
Collecting terms,
\[
	A(z) \left[ 1 - 3z + 3z^2 - z^3 \right] = z - 2z^2
\]
\[
	A(z) = \frac{z-2z^2}{1-3z+3z^2-z^3} \\
\]
By partial fraction expansion,
\begin{align*}
	A(z) &= \frac{2}{z-1} + \frac{3}{(z-1)^2} + \frac{1}{(z-1)^3} \\
		&= \frac{-2}{(1-z)} + \frac{3}{(1-z)^2} - \frac{1}{(1-z)^3}
\end{align*}

Extract coefficients for a closed for for $a_n$:
\begin{align*}
	a_n &= \left[z^n\right] \frac{-2}{(1-z)} + \left[z^n\right] \frac{3}{(1-z)^2} - \left[z^n\right] \frac{1}{(1-z)^3} \\
		&= -2 + 3(n+1) - \frac{(n+1)(n+2)}{2} \\
		&= \frac{(3-n)n}{2} 
\end{align*}

\section{Derivation of Formula for Catalan Numbers}
Catalan numbers satisfy the recurrence
\[
	T_N = \sum_{0 \le k < N} T_k T_{N-k-1} + \delta_{N0}
\]
Form the generating function by applying the definition $T(z) = \sum T_k z^k$:
\[
	T(z) = \sum_{N \ge 0} T_N z^N = \sum_{N \ge 0} \sum_{0 \le k < N} T_K T_{N-k-1} z^N + 1
\]
Change the order of summation:
\[
	T(z) = 1 + \sum_{k \ge 0} \sum_{N>k} T_k T_{N-k-1} z^N
\]
Replace $N$ by $N+k+1$ to clean up the inner summation limit: $N>k$ becomes $N+k+1 > k$ or $N+1>0$ or $N \ge 0$.
\[
	T(z) = 1 + \sum_{k \ge 0} \sum_{N \ge 0} T_k T_N z^{N+k+1}
\]
Distribute powers of $z$:
\[
	T(z) = 1 + z \left(\sum_{k \ge 0} T_k z^k\right) \left( \sum_{N \ge 0} T^N z^N \right) = 1 + z T^2(z)
\]
This is a quadratic expression in $T(z)$:
\[
	zT^2(z) - T(z) + 1 = 0
\]
Apply the quadratic formula with $a=z$, $b=-1$, $c=1$:
\[
	T(z) = \frac{1 \pm \sqrt{(1-4z)}}{2z}
\]
By initial conditions, use the negative formulation. (WHY?) Rearrange to
\[
	zT(z) = -\frac{1}{2}\left(\sqrt{1-4z} -1\right)
\]
Treat $\sqrt{1-4z}$ as a binomial expression $(a+b)^m$ with $a=1$, $b=-4z$ and $m=\frac{1}{2}$. By the binomial theorem,
\[
	\sqrt{1-4z} =  \sum_{N\ge0} \binom{m}{N} a^{m-N} b^N =  \sum_{N\ge0} \binom{\frac{1}{2}}{N} 1^{(\frac{1}{2}-N)} (-4z)^N = 1+ \sum_{N>0} \binom{\frac{1}{2}}{N} (-4)^N z^N
\]
\[
	zT(z) = -\frac{1}{2}\sum_{N>0} \binom{\frac{1}{2}}{N} (-4)^N z^N
\]
Equate coefficients of the polynomials:
\[
	\left[ z^N \right] zT(z) = T_{N-1} = \frac{1}{2} \binom{\frac{1}{2}}{N} (-4)^N
\]
or
\[
	T_N = -\frac{1}{2} \binom{\frac{1}{2}}{N+1} (-4)^{N+1}
\]
By the falling factorial definition of a binomial coefficient,
\[
	T_N = -\frac{1}{2} \frac{(\frac{1}{2})(\frac{1}{2}-1)(\frac{1}{2}-2)(\frac{1}{2}-3)\cdots(\frac{1}{2}-N)}{(N+1)!} (-4)^{N+1}
\]
where the numerator has $N+1$ factors in the product.  Multiply all but the first of those factor by -2:
\[
	T_N = -\frac{1}{2} \frac{(\frac{1}{2})1 \cdot 3 \cdot 5 \cdots (2N-1)}{(N+1)!} 2^N (-4)
\]
Cancel the $-4$ and the $\frac{1}{2}$ and $-\frac{1}{2}$, leaving $N$ factors in the numerator
\[
	T_N = \frac{1 \cdot 3 \cdot 5 \cdots (2N-1) }{(N+1)!} 2^N
\]
Add in the missing even terms in numerator and denominator:
\[
	T_N = \frac{1 \cdot 3 \cdot 5 \cdots (2N-1) }{(N+1)!} \frac{2 \cdot 4 \cdot 6 \cdots 2N}{2 \cdot 4 \cdot 6 \cdots 2N} 2^N
\]
Cancel the remaining powers of 2:
\begin{align*}
	T_N &= \frac{1 \cdot 3 \cdot 5 \cdots (2N-1)}{(N+1)!}  \frac{2 \cdot 4 \cdot 6 \cdots 2N}{1 \cdot 3 \cdot 5 \cdots N} \\
		&= \frac{1}{N+1} \frac{(2N)!}{N! N!} \\
		&= \frac{1}{N+1} \binom{2N}{N}
\end{align*}
\end{document}  